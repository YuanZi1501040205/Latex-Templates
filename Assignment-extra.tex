%----------------------------------------------------------------------------------------
%	Modified class referred tex document.
%----------------------------------------------------------------------------------------

\documentclass[layout=rotate,useSpeTable,useQuote,useParallel,useLettrine]{CKAgn}
\ckasetup{ 
  type={Math},
  author={Yuchen Jin},
  organization={Test the organization},
  number={1},
  textStyle={box}
}
\usepackage{metalogo}
\usepackage{lipsum}

\begin{document}

\maketitle % Print the title

\section{Introduction}

In this example, we show the extra functions of this class.

\section{Examples}

\subsection{Example of special table}

The features in this sub-sections requires the option \texttt{useSpeTable}.

In \autoref{tab:Table1}, we would add some descriptions for each row.

\begin{table}[htbp]
  \centering
  \caption{Example table \label{tab:Table1}}
  \begin{tabular}{|m{0.1\columnwidth}<{\centering}|m{0.46\columnwidth}|}
    \hline
    \textbf{Symbol} & \makebox[0.46\columnwidth][c]{\textbf{Description}} \\ \hline
    $\alpha$ & text 1 \fullrow{In this part, we show some extra descriptions for the first row.} \\ \hline
    $\Gamma$ & text 2 \fullrow{In this part, we show some extra descriptions for the second row.} \\ \hline
    $\Omega$ & text 3 \fullrow[2]{In this part, we show some extra descriptions for the third row. The column number is explicitly assigned as 2.} \\ \hline
  \end{tabular}
\end{table}

In the following environment, we show some examples of word boxes.
\vspace{3em}
\begin{wordlist}
  \begin{wordbox}{benefit our lives}{vi.}
    \witem{g}{= improve our lives}
    \witem{w}{\weg[benefited our lives]{The internet has benefited our lives.}}
  \end{wordbox}
  \begin{wordbox}{doorstep}{n.}
    \witem{g}{a step leading up to the outer door of a house.}
    \witem{w}{\weg{What we buy get delivered to our doorstep.}}
  \end{wordbox}
  \begin{wordbox}{mail a letter}{vi.}
    \witem{g}{send a paper-based letter.}
    \witem{w}{\weg{We mail a letter in the past.}}
  \end{wordbox}
  \begin{wordbox}{enjoyable}{adj.}
    \witem{g}{an adjective that describes anything full of delight and fun.}
    \witem{w}{\weg{It is not enjoyable to} \blank.}
  \end{wordbox}
  \begin{wordbox}{take \blank for example}{v.}
    \witem{g}{=for example, \blank}
  \end{wordbox}
  \begin{wordbox}{class schedule}{n.}
    \witem{g}{the arrangement of classes}
    \witem{w}{\weg[class schedules]{All my class schedules are} \emph{on} my phone.}
  \end{wordbox}
  \begin{wordbox}{enormously}{adv.}
    \witem{g}{= greatly}
    \witem{w}{\weg{The cellphone helps enormously.}}
  \end{wordbox}
  \begin{wordbox}{stay in touch with}{vt.}
    \witem{g}{= keep in touch with \blank}
    \witem{w}{\weg{We could stay in touch with our loved ones.}}
  \end{wordbox}
  \begin{wordbox}{split}{vt.}
    \witem{g}{divide into two or more groups.}
    \witem{w}{\weg{Living together allows us to split the rent, utility and food costs.}}
  \end{wordbox}
  \begin{wordbox}{a bunch of}{adj.}
    \witem{g}{a group of \blank}
    \witem{w}{\weg{a bunch of people}}
  \end{wordbox}
  \begin{wordbox}{collect one's thought}{vi.}
    \witem{g}{= think}
    \witem{w}{\weg[collect their thoughts]{The students could have a break away from their busy life and collect their thoughts.}}
  \end{wordbox}
  \begin{wordbox}{be more productive}{vi.}
    \witem{g}{= work more efficiently}
    \witem{w}{\weg{I could be more productive if staying alone.}}
  \end{wordbox}
  \begin{wordbox}{block out}{vt.}
    \witem{g}{= keep (ourselves) away from}
    \witem{w}{\weg{Music could block out the noise in the working staff.}}
  \end{wordbox}
  \begin{wordbox}{s}{}
    \witem{gb}{I can't even imagine what my life would be like if \blank.}
    \witem{w}{\weg[I can't even imagine what my life would be like if]{I can't even imagine what my life would be like if music and movies don't exist.}}
  \end{wordbox}
  \begin{wordbox}{a record of history}{n.}
    \witem{g}{= historical relics}
    \witem{w}{\weg{They leave a record of history to us.}}
  \end{wordbox}
  \begin{wordbox}{be involved in}{vt.}
    \witem{g}{= participate in \blank}
    \witem{w}{\weg{Tony starts to be involved in many activities.}}
  \end{wordbox}
  \begin{wordbox}{have the same taste in}{vt.}
    \witem{g}{= have the same interests in \blank}
    \witem{w}{\weg{Tony and I have the same taste in music and art.}}
  \end{wordbox}
  \begin{wordbox}{collaborate with}{vt.}
    \witem{g}{= cooperate with \blank}
    \witem{w}{\weg{I get more done when I collaborate with a team.}}
  \end{wordbox}
  \begin{wordbox}{bond}{vt.}
    \witem{g}{= connect}
    \witem{w}{\weg{Reading could bonds people like no other.}}
  \end{wordbox}
  \begin{wordbox}{First off}{adv.}
    \witem{g}{= Firstly/First}
    \witem{w}{\weg{First off, it will allow you to} \blank.}
  \end{wordbox}
\end{wordlist}

\subsection{Example of quote}

The features in this sub-sections requires the option \texttt{useQuote}.

Here we show 3 examples:

\begin{shadeQuote}{}
  A quote without source.
\end{shadeQuote}

\begin{shadeQuote}{Shakespeare}
  To be, or not to be, that is the question.
\end{shadeQuote}

\begin{shadeQuote}[r]{Shakespeare}
  To be, or not to be, that is the question.
\end{shadeQuote}

\subsection{Example of parallel}

The parallel environment is usually used for writing translations. The features in this sub-sections requires the option \texttt{useParallel}. We also show examples of \texttt{$\backslash$textfoot}.

\begin{Parallel}[v]{0.42\textwidth}{0.52\textwidth}
  \litem{\lipsum[2]}
  \ritem{\textfoot{dd}{xxxx}2009, MIT opening course project had\textfoot{presented}{ present/release/publish}about approximately 2000 courses, which can be accessed freely online by people around the world. However, there are still some, if not many, are of the\textfoot{opinion}{or believing that}that when it comes to the education, the traditional educational methods has irreplaceable advantages when compared to the other\textfoot{methods}{approach/method/mode}. Personally I tend to think that this conclusion cannot hold its ground.}
  \litem{\lipsum[4]}
  \ritem{\lipsum[5]}
\end{Parallel}

\subsection{Example of lettrine}

\lettrine{I}{n} this paragraph, we show an example of the paragraph with the first letter emphasized by the \texttt{lettrine} package. This feature requires the option \texttt{useLettrine}.

\end{document} 